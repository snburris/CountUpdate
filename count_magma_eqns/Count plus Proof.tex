\documentclass[12pt]{amsart}
\usepackage{amsmath}
\usepackage{amssymb}   % has \smallsetminus
\usepackage{mathabx}    %%% has \coloneq

\renewcommand{\baselinestretch}{1.1}

\newcommand{\Invol}{{\sf Invol}}
\newcommand{\Dd}{{\,\diamond\,}}
\newcommand{\Eqns}{{\sf Eqns}}
\newcommand{\eq}{{\;\approx\;}}    

\title{Counting Magma Equations}

\date{\today}

\begin{document}
\maketitle

\section{The formula}
The symbol $\Dd$ will be used for {\it the binary operation} of magma equations. 
The {\it order of a magma term} is the number of occurrences of $\Dd$ in the term. 
Thus the order of $(x \Dd y) \Dd (x \Dd z)$ is 3. 
There is an obvious mapping from terms of order $n$ to plane binary trees of order $n$. 
(Additionally the leaves can be naturally labelled by the variables in the term to make the mapping a
bijection.)
Let
$$
T(n) = \frac{1}{n+1} \cdot {2n \choose n} \ .
$$
This counts the number of plane binary trees of order $n$.

The {\it order of a magma equation} is the sum of the orders of the two sides. 
Thus the order of $(x \Dd y)  \Dd (x \Dd z) \approx z  \Dd  (x  \Dd x)$ is 5. 
The equation $(v \Dd w)  \Dd (v \Dd u) \approx u  \Dd  (v  \Dd v)$ obtained by relabelling the variables is considered to be ``the same'' as the previous equation as far as counting is concerned, as is the equation 
$ z  \Dd  (x  \Dd x) \approx (x \Dd y)  \Dd (x \Dd z) $
obtained by switching the two terms. 

The {\it Stirling numbers $n \brace m$ of the second kind} count the number of ways to partition a set of $n$ elements into $m$ non-empty subsets.

The number of bijections $\beta$ of a set of $n$ elements to itself is $n!$ .
The number of involutions $\beta$ (that is, $\beta = \beta^{-1}$) is given by
$$
\Invol(n) = \sum_{0 \leq k \leq \lfloor n/2 \rfloor}  {n\choose 2k} (2k-1)!!.
$$

For non-negative integers $a_1, a_2$ let $\fbox{$E(a_1,a_2)$}$ be the number of magma equations 
$t_1 \approx t_2$ with $t_i$ of order $a_i$, counting up to relabelling, up to switching terms, and only allowing the equation $x\approx x$ of the form $t \approx t$. Then the following holds:

\pagebreak

$\bullet$ $E(0,0) = 2$.\\

$\bullet$ If \fbox{$a_1\neq a_2$}: 
$$
E(a_1,a_2) = T(a_1)\cdot T(a_2)\cdot \sum_{\substack{1 \le p_i \le a_i+1\\0\le s \le \min(p_1,p_2)}} 
{a_1+1 \brace p_1} {a_2+1 \brace p_2} {p_1 \choose s} {p_2 \choose s} s! \ .
$$

$\bullet$ If \fbox{$a_1= a_2 = a>0$}: 
\begin{align*}
E(a,a) =\ &\frac{1}{2} T(a)^2\cdot \sum_{\substack{1 \le p_1,p_2 \le a+1\\ 0\le s \le \min(p_1,p_2)}} 
{a+1 \brace p_1} {a+1 \brace p_2} {p_1 \choose s} {p_2 \choose s} s! \\
&+ \frac{1}{2} T(a) \cdot \sum_{\substack{1 \le p \le a+1\\ 0\le s \le p}} 
{a+1 \brace p}  {p \choose s}  \Invol(s)\\
&\quad - T(a)\cdot \sum_{1\le p \le a+1}  {a+1 \brace p} \ .
\end{align*}

Let $\Eqns(n)$ be the number of magma equations of order $n$ (under the given constraints). Then
$$
\Eqns (n) = \sum_{0 \le a \le \lfloor n/2 \rfloor } E(a,n-a) .
$$

Let $\Eqns^\star(n)$ be the number of magma equations of order $\le n$. Then
$$
\Eqns^\star(n) = \sum_{0 \le k \le n} \Eqns (k) .
$$

{\bf Maple calculations} for $E(a_1,a_2)$ with $0\le a_1 \le 2$, $0 \le a_2 \le 5$, and for
$\Eqns(n)$, $\Eqns^\star(n)$ with $0\le n \le 5$:
\bigskip

$
\begin{array}{ r | r  r  r  r  r  r}
E& 0 & 1 & 2 & 3 & 4 & 5\\
\hline
0& 2 & 5 & 30 & 260 & 2842 & 36834\\
1&  & 9 & 104 & 1015 & 12278 & 173880\\
2&  &  & 427 & 8770 & 115920 & 1776348\\
\end{array}
$
\qquad
$
\begin{array} {r r r}
n &\Eqns(n) & \Eqns^\star(n)\\
\hline
0 & 2 & 2\\
1 & 5 & 7\\
2 & 39 & 46\\
3 & 364 & 410\\
4 & 4284 & 4694\\
5 & 57882 & 62576
\end{array}
$
\newpage

\section{The Proof}

A {\em frame} $\varphi$ is a 7-tuple $(T_1,\pi_1,S_1, \beta, T_2, \pi_2, S_2)$ where 
\begin{itemize} 
\item the $T_i$ are plane binary trees, 
\item $\pi_i$ is a partition of the set $\{1,\ldots,a_i + 1\}$ where $a_i$ is the 
order of $T_i$ which is $(|T_i| -1)/2$,
\item $S_i$ is a subset of $\{1,\ldots,p_i\}$ where $p_i = |\pi_i|$, the number of blocks in $\pi_i$,
\item $|S_1| = |S_2|$, and
\item $\beta : S_1 \rightarrow S_2$ is a bijection.
\end{itemize}

The frame $\Phi(t_1 \approx t_2)$ of a magma equation $t_1 \approx t_2$ is defined as follows:
\begin{itemize}
\item $T_i$ is the plane binary tree associated with $t_i$.
\item The variables of $t_i$ have locations $1,\ldots, a_i+1$, counting from the left.
\item $\pi_i$ is the partition of the set $\{1,\ldots,a_i + 1\}$ defined by: \\
$j$ and $k$ are in
the same block of $\pi_i$ iff the variables of $t_i$ at the locations $j$ and $k$ are the same.

The variable associated to a block of $\pi_i$ is the variable at all the locations in the block.
\item 
Number the blocks of $\pi_i$ by $1,\ldots,p_i$ such that for $1 \le j<k \le p_i$ the least member of the
$j$th block is less than the least member of the $k$th block.
\item
$S_i$ is the set of block numbers from $\{1,\ldots,p_i\}$ whose associated variables in $t_i$ appear 
on both sides of the equation.
\item
For $j \in S_1$, $\beta(j)$ is the number of the block in $S_2$ whose associated variable is the same as that associated to block number $j$ in $S_1$. 
\end{itemize}

{\bf FACT:} Two magma equations $t_1 \approx t_2$ and $t_1' \approx t_2'$ are the same up to a 
relabelling of the variables iff their frames are equal.\\

{\bf FACT:} Every frame $\varphi$ is the frame of a magma equation.\\

These facts allow us to compute the number of magma equations up to a relabelling of the variables 
by counting frames.

Let $\Phi(a_1,a_2)$ be the collection of frames with the order of $T_i$ equal $a_i$. 
Then, up to relabelling, the number of magma equations $t_1 \approx t_2$ 
with order($t_i$) $= a_i$ is $|\Phi(a_1,a_2)|$, 
which is
\begin{equation}\label{a1noteqa2}
T(a_1)\cdot T(a_2)\cdot \sum_{\substack{1 \le p_1 \le a_1+1\\ 1 \le p_2 \le a_2+1\\0\le s \le \min(p_1,p_2)}} 
{a_1+1 \brace p_1} {a_2+1 \brace p_2} {p_1 \choose s} {p_2\choose s} s! \ .
\end{equation}

Counting equations up to relabelling and symmetry means that one treats
$t_1 \approx t_2$ and $t_2 \approx t_1$ as a single equation.
This means that when their frames are different only one frame is counted. 
To accomplish this when counting all equations of order $n$ (that is, when $a_1 + a_2 = n$) 
the first step is simply to count
only one of $\Phi(a_1,a_2)$ and $\Phi(a_2,a_1)$ when $a_1 \neq a_2$, say only count $\Phi(a_1,a_2)$
 when $a_1<a_2$. 

The case $a_1=a_2 = a$, which only occurs when $n$ is even, requires more consideration since
the frames of $t_1 \approx t_2$ and $t_2 \approx t_1$ are both in $\Phi(a,a)$, and sometimes
they are the same, sometimes they are distinct.
 Given $$\Phi(t_1 \approx t_2) = (T_1,\pi_1,S_1, \beta, T_2, \pi_2, S_2)$$ one has
 $$\Phi(t_2 \approx t_1) = (T_2, \pi_2, S_2, \beta^{-1}, T_1, \pi_1, S_1).$$
 The two equations have the same frame iff 
 $$T_1 = T_2,\  \pi_1 = \pi_2,\  S_1=S_2,\ \beta = \beta^{-1}.$$ 
 This means $\beta$ is an involution. This leads to the count,
 up to relabelling and symmetry, of
 equations with frames in $\Phi(a,a)$ being
 \begin{align*}
&\frac{1}{2} T(a)^2\cdot \sum_{\substack{1 \le p_1,p_2 \le a+1\\ 0\le s \le \min(p_1,p_2)}} 
{a+1 \brace p_1} {a+1 \brace p_2} {p_1 \choose s} {p_2 \choose s} s! \\
&+ \frac{1}{2} T(a) \cdot \sum_{\substack{1 \le p \le a+1\\ 0\le s \le p}} 
{a+1 \brace p}  {p \choose s}  \Invol(s) \ .
\end{align*}
To avoid trivial equations $t\approx t$, from this one needs to subtract
$$T(a)\cdot \sum_{1\le p \le a+1}  {a+1 \brace p} \ .$$
To then include one trivial equation in the count, namely $x\approx x$, only use this subtraction
for $a>0$.
\end{document}

